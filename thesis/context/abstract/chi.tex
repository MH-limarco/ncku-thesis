% ------------------------------------------------
\StartAbstractChi
% ------------------------------------------------

圖神經網路(Graph Neural Networks, GNNs)於多種圖結構資料分析任務中表現卓越。然而,其在實務應用上仍面臨三大環環相扣的挑戰:資料不完整、高昂的訓練成本,以及過高的推論開銷。儘管現有方法各自處理單一問題,卻未能提供一個整體的解決方案。

為解決此問題,本研究提出 \textbf{{\framework}}({\frameworkname}),一個建立在「\textbf{協同優化}」核心原則之上的統一框架,旨在對 GNN 的完整流程進行聯合優化。本框架透過一套連貫的流程來實現此設計哲學:首先,以高效的特徵補全技術確保資料的完整性;接著,採用一創新的代理剪枝策略,打造出一個輕量且強健的 GNN 教師模型;最終,再將其知識蒸餾至一個更高效的 MLP 學生模型中,以達成快速推論的目的。

大量的實驗證明,{\framework}不僅能在高達 99.5\% 特徵缺失的資料集上保持高準確性,其\textbf{推論速度更比基準 GNN 教師模型提升超過五倍}。本研究證明,此一整體性的設計方法,為建構能夠實際應用於真實世界、兼具魯棒性、高效率與可部署性的 GNN,提供了一條實用且強而有力的途徑。

% ------------------------------------------------
\EndAbstractChi
% ------------------------------------------------
