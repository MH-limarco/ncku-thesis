% 定義底色
\definecolor{lightgray}{gray}{0.95}
\definecolor{blue1}{RGB}{0,128,255}
\definecolor{orange1}{RGB}{235,120,23}
\definecolor{green1}{RGB}{20,170,70}
\definecolor{mygray}{gray}{0.825}


\begingroup
\rowcolors{2}{mygray}{white}
%\renewcommand{\arraystretch}{1.12} % 增加行高,表格更寬鬆

\begin{tabular}{L{0.16\textwidth}|C{0.15\textwidth}C{0.175\textwidth}L{0.21\textwidth}L{0.21\textwidth}}
\toprule
\textbf{Method} & \textbf{\small Imputation Performance} & \textbf{\small Computational Cost} & \textbf{\small Key Idea} & \textbf{\small Key Limitation} \\
\midrule
Zero/Mean & Low & Very Low & Constant Filling & Unable to capture characteristic distribution \\
LP~\cite{LP} & Low/Medium & Very Low & Feature Replacement & Ignore existing features \\
PaGNN~\cite{PaGNN} & Medium & Very High & Learnable Method & Sensitive to noise, poor scalability \\
GCNMF~\cite{GCNMF} & Medium & Very High & Matrix Factorization, Learnable Method & Sensitive to noise, poor scalability \\
FP~\cite{FP} & Medium & Low & Feature Propagtion-based & Insensitive to differences between features\\
PCFI~\cite{PCFI} & High & High & Confidence-aware Feature Propagtion & Channel-by-channel calculation, poor scalability \\
\textbf{APCFI(Ours)} & \textbf{High} & \textbf{Low} & Approximate Confidence \& Joint Channel Diffusion & Not applicable to heterogeneous graphs\\

\midrule
\end{tabular}
\caption{\textbf{Comparison of representative feature imputation methods.}}
\label{tab:method_imputaion}
\endgroup