% ------------------------------------------------
\StartChapter{Related Work}{chapter:related-work}
% ------------------------------------------------

This chapter reviews prior research in areas crucial to our work.
Methods for handling incomplete data in GNNs, particularly feature imputation, are first reviewed.
The subsequent sections discuss the landscape of GNN model pruning and knowledge distillation.
For each area, the limitations of existing approaches are analyzed to precisely situate the contributions of the proposed iPaD framework.

% --- Section 2.1: Feature Imputation ---
\section{Feature Imputation in Graph Neural Networks}
\label{sec:imputation}
Feature imputation is a critical component for enhancing the robustness of Graph Neural Networks (GNNs) in the presence of missing node features---a common challenge in real-world graph-structured data~\cite{you2020handling, FP, GCNMF, PCFI}.
Existing approaches can be broadly classified into heuristic methods, learning-based models, and feature propagation techniques.


Heuristic methods, such as Zero, Mean, or Label Propagation (LP) Imputation, are computationally efficient but often fail to capture the complex feature distributions present in graphs~\cite{LP}, leading to suboptimal performance.
Learning-based approaches, including GCNMF and PaGNN, employ deep neural network architectures to model and reconstruct missing features~\cite{PaGNN, GCNMF}.
Despite their theoretical expressiveness, these models can be limited by substantial computational overhead and may not outperform simpler methods in high-missingness regimes.


More recently, non-learning-based feature propagation methods have gained traction.
Feature Propagation (FP) exploits the underlying graph structure to propagate known features, offering a strong balance of performance and efficiency~\cite{FP}.
Pseudo-Confidence Feature Imputation (PCFI) further advances this by introducing channel-wise pseudo-confidence to guide a two-stage diffusion process,
achieving state-of-the-art imputation quality~\cite{PCFI}.
However, this high performance comes at a significant cost: PCFI's channel-by-channel calculation leads to extremely high computational complexity, hindering its scalability to large graphs.
This reveals a critical trade-off between imputation performance and computational efficiency in the current literature.


To address this efficiency-accuracy trade-off,
\textbf{APCFI} is introduced to retain the high imputation quality of confidence-aware methods such as PCFI, while drastically reducing computational burden via Approximate Shortest Distance Estimation (ASDE) and Dynamic Joint Channel Diffusion (DJCD).

Table~\ref{tab:method_imputaion} provides a clear comparison of these imputation strategies.


\begin{table}[htbp]
    \centering
    % 定義底色
\definecolor{lightgray}{gray}{0.95}
\definecolor{blue1}{RGB}{0,128,255}
\definecolor{orange1}{RGB}{235,120,23}
\definecolor{green1}{RGB}{20,170,70}
\definecolor{mygray}{gray}{0.825}


\begingroup
\rowcolors{2}{mygray}{white}
%\renewcommand{\arraystretch}{1.12} % 增加行高,表格更寬鬆

\begin{tabular}{L{0.16\textwidth}|C{0.15\textwidth}C{0.175\textwidth}L{0.21\textwidth}L{0.21\textwidth}}
\toprule
\textbf{Method} & \textbf{\small Imputation Performance} & \textbf{\small Computational Cost} & \textbf{\small Key Idea} & \textbf{\small Key Limitation} \\
\midrule
Zero/Mean & Low & Very Low & Constant Filling & Unable to capture characteristic distribution \\
LP~\cite{LP} & Low/Medium & Very Low & Feature Replacement & Ignore existing features \\
PaGNN~\cite{PaGNN} & Medium & Very High & Learnable Method & Sensitive to noise, poor scalability \\
GCNMF~\cite{GCNMF} & Medium & Very High & Matrix Factorization, Learnable Method & Sensitive to noise, poor scalability \\
FP~\cite{FP} & Medium & Low & Feature Propagtion-based & Insensitive to differences between features\\
PCFI~\cite{PCFI} & High & High & Confidence-aware Feature Propagtion & Channel-by-channel calculation, poor scalability \\
\textbf{APCFI(Ours)} & \textbf{High} & \textbf{Low} & Approximate Confidence \& Joint Channel Diffusion & Not applicable to heterogeneous graphs\\

\midrule
\end{tabular}
\caption{\textbf{Comparison of representative feature imputation methods.}}
\label{tab:method_imputaion}
\endgroup
\end{table}

% --- Section 2.2: Model Pruning ---
\section{Model Pruning in Graph Neural Networks}
\label{sec:pruning}
As GNNs scale to handle increasingly large and complex graphs, model pruning has emerged as a crucial strategy to reduce computational costs during both training and inference.
Existing GNN pruning paradigms include channel pruning (e.g., GCNP~\cite{GCNP}), hardware-aware optimization (e.g., PruneGNN~\cite{pruneGNN}),
and methods that simultaneously prune graph structures and model weights during training (e.g., UGS~\cite{UGS}, ICPG~\cite{ICPG}).


However, these methods often suffer from two major limitations.
First, they typically entangle the pruning process with the GNN training pipeline,
introducing significant computational overhead and complex, model-specific algorithmic designs that can substantially \textbf{increase the total training cost}.
Second, they are often proposed as specific, monolithic algorithms, lacking the \textbf{scalability and flexibility} to incorporate a wide array of mature pruning techniques from other domains, such as the MLP literature.

To overcome these challenges of high overhead and low flexibility, we introduce \textbf{MPP} (Mirror Projection Pruning).
Instead of being another monolithic algorithm, MPP proposes a new \textbf{proxy-based pruning paradigm}.
Its core idea is to decouple pruning from GNN training by projecting the GNN to an isomorphic MLP, where mature, off-the-shelf pruning algorithms can be applied efficiently \textit{before} the main training commences.
The advantages of this paradigm are summarized in Table~\ref{tab:method_pruning}.

\begin{table}[htbp]
    \centering
    % 定義底色
\definecolor{lightgray}{gray}{0.95}
\definecolor{blue1}{RGB}{0,128,255}
\definecolor{orange1}{RGB}{235,120,23}
\definecolor{green1}{RGB}{20,170,70}
\definecolor{mygray}{gray}{0.825}


\begingroup
\rowcolors{2}{mygray}{white}
\renewcommand{\arraystretch}{1.25} % 增加行高,表格更寬鬆

\begin{tabular}{L{0.175\textwidth}|C{0.2\textwidth}C{0.125\textwidth}L{0.375\textwidth}}

\toprule
\textbf{\small Method} & \textbf{\small Total Training Cost} & \textbf{\small Scalability} & \textbf{\small Key Idea} \\
\midrule
GCNP~\cite{GCNP} & \uparrow & {\XSolidBrush} & Using a novel LASSO regression formulation to prune model channels \\
PruneGNN~\cite{pruneGNN} &  \uparrow & {\XSolidBrush} & Introduces novel SIMD-aware kernels and accelerate structured GNN pruning on parallel hardware \\
UGS~\cite{UGS} & \uparrow & {\XSolidBrush} & Prune graphs and weights simultaneously during training for specific ensemble algorithms \\
ICPG~\cite{ICPG} & \uparrow & {\XSolidBrush} & Iterative co-pruning during training for specific ensemble algorithms \\
\textbf{MPP(Ours)} & \downarrow & {\CheckmarkBold} & A GNN proxy pruning framework that off-the-shelf MLP-compatible algorithms \\

\midrule
\end{tabular}
\caption{\textbf{Comparison of representative GNN pruning methods.}}
\label{tab:method_pruning}
\endgroup
\end{table}

% --- Section 2.3: Knowledge Distillation ---
\section{Knowledge Distillation for Graph Models}
\label{sec:distillation}
Knowledge distillation (KD) is a pivotal technique for compressing GNNs, enabling the deployment of lightweight yet high-performing models.
In the context of graph data, KD approaches can be broadly categorized into GNN-to-GNN and GNN-to-MLP distillation.


GNN-to-GNN methods, such as LSP~\cite{LSP} and TinyGNN~\cite{TinyGNN}, focus on transferring knowledge from a large GNN teacher to a smaller GNN student.
While effective, they do not eliminate the graph dependency at inference time. To achieve maximum inference speed, GNN-to-MLP distillation has emerged as a promising direction. Methods like GLNN~\cite{GLNN} show that an MLP student, once distilled, can achieve competitive performance without access to the graph structure. More advanced techniques like KRD~\cite{KRD} introduce reliability-aware curriculum learning to handle potentially noisy knowledge from the teacher.


Despite these advances, existing GNN-to-MLP methods exhibit several gaps.
First, the student MLP is typically an \textbf{arbitrary MLP}, lacking any intrinsic structural alignment with the GNN teacher. Second, the distillation process is designed \textbf{independently of any model pruning stage}, missing a key opportunity for synergistic optimization. Finally, while some methods address teacher noise, a more robust, pipeline-aware approach is needed.


Our proposed \textbf{MP-KRD} (Mirror Projection Knowledge-inspired Reliable Distillation) is designed specifically to fill these gaps.
It creates a powerful synergy by using the \textbf{structurally-aligned, isomorphic MLP} generated by our MPP stage as the student. This ensures architectural consistency and maximizes the effectiveness of the knowledge transfer, which is further enhanced by a reliability-aware curriculum.
The unique characteristics of MP-KRD are summarized in Table~\ref{tab:method_kd}.

\begin{table}[htbp]
    \centering
    % 定義底色
\definecolor{lightgray}{gray}{0.95}
\definecolor{blue1}{RGB}{0,128,255}
\definecolor{orange1}{RGB}{235,120,23}
\definecolor{green1}{RGB}{20,170,70}
\definecolor{mygray}{gray}{0.825}


\begingroup
\rowcolors{2}{mygray}{white}
%\renewcommand{\arraystretch}{1.25} % 增加行高,表格更寬鬆

\begin{tabular}{L{0.2\textwidth}|C{0.2\textwidth}C{0.125\textwidth}C{0.125\textwidth}L{0.25\textwidth}}

\toprule
\textbf{\small Method} & \textbf{\small Student Source} & \textbf{\small {Feature Alignment}} & \textbf{\small {Handles Noisy Teacher}} & \textbf{\small Key Idea} \\
\midrule
LSP~\cite{LSP} & Small GNN & {\XSolidBrush} & {\XSolidBrush} & Transfer the topological structure \\
GNN-SD~\cite{GNNSD} & Teacher GNN & \textbf{\CheckmarkBold} & {\XSolidBrush} & GNNs Self-Distillation framework \\
TinyGNN~\cite{TinyGNN} & Small GNN & {\XSolidBrush} & {\XSolidBrush} &  Distillation with fewer parameter GNNs\\
GLNN~\cite{GLNN} & Arbitrary MLP & {\XSolidBrush} & {\XSolidBrush} & Learning predictions from GNNs \\
KRD~\cite{KRD} & Arbitrary MLP & {\XSolidBrush} & {\CheckmarkBold} & Reliability-based learning \\
\textbf{MP-KRD(Ours)} &  \textbf{Isomorphic MLP} & \textbf{\CheckmarkBold} & \textbf{\CheckmarkBold} & Structural Alignment + Reliability Course Learning \\

\midrule
\end{tabular}
\caption{\textbf{Comparison of representative knowledge distillation methods for graph models.}}
\label{tab:method_kd}
\endgroup
\end{table}


% --- Section 2.4: SOTA GNNs ---
\section{High-Level Comparison of SOTA GNNs}
\label{sec:rw_sota_comparison}

Beyond the specific areas of imputation and compression, it is also instructive to compare our approach against the broader landscape of recent state-of-the-art (SOTA) GNNs, particularly concerning the trade-off between performance and efficiency.

Table~\ref{tab:method_sota} provides a qualitative comparison of prominent SOTA models. As the table illustrates, while most recent Transformer-based models like GraphGPS\cite{GPS} and NAGphormer\cite{NAG} achieve high predictive performance, they often do so at the cost of low computational efficiency. Our \textbf{MP-TunedGNN} (the teacher model in our framework), by integrating the MPP module, stands out by maintaining high performance while significantly boosting efficiency compared to both its ``TunedGNN'' backbone and other SOTA models. This highlights a key contribution of our work in addressing the prevalent performance-efficiency trade-off in modern GNN architectures.
\begin{table}[htbp]
    \centering
    \input{./context/2-related-work/_sota_comparison_table}
\end{table}

% --- Section 2.5: Chapter Summary ---
\section{Chapter Summary}
\label{sec:rw_summary}
In this chapter, we have reviewed prior art across feature imputation, model pruning, and knowledge distillation, identifying the key limitations of existing methods in each domain.
To synthesize these findings and crystallize the motivation for our work, we present a holistic comparison in Table~\ref{tab:holistic_comparison}.
This table evaluates a wide spectrum of methods against the three core challenges identified in this thesis: handling data \textbf{Completeness}, achieving high \textbf{Training Efficiency}, and ensuring model \textbf{Scalability} for inference.

The table clearly illustrates the``siloed'' nature of existing research.
Imputation methods like PCFI excel at ``Completeness'' but do not address efficiency.
High-performance SOTA models like GraphGPS~\cite{GPS} have low ``Training Efficiency'' and ``Scalability''.
Distillation methods like KRD offer ``Scalability'' but do not handle data ``Completeness''.

\begin{table}[htbp]
    \centering
    % 定義底色
\definecolor{lightgray}{gray}{0.95}
\definecolor{blue1}{RGB}{0,128,255}
\definecolor{orange1}{RGB}{235,120,23}
\definecolor{green1}{RGB}{20,170,70}
\definecolor{mygray}{gray}{0.825}


\begingroup
\rowcolors{2}{mygray}{white}
\renewcommand{\arraystretch}{1.3} % 增加行高,表格更寬鬆

\begin{tabular}{L{0.25\textwidth}|C{0.2\textwidth}C{0.125\textwidth}C{0.125\textwidth}}

\toprule
\textbf{\small Method} & \textbf{\small Completeness} & \textbf{\small {Training Efficiency}} & \textbf{\small {Scalability}}\\
\midrule
PaGNN~\cite{PaGNN} &  {\CheckmarkBold} & - & - \\
GCNMF~\cite{GCNMF} &  {\CheckmarkBold} & - & - \\
FP~\cite{FP} & {\CheckmarkBold} & - & - \\
PCFI~\cite{PCFI} & {\CheckmarkBold} & - & - \\
\midrule
GraphGPS~\cite{GPS} & - & {\CheckmarkBold} & -\\
NAGphormer~\cite{NAG} & - & {\CheckmarkBold} & -\\
Exphormer~\cite{Exphormer} & - & {\CheckmarkBold} & -\\
GOAT~\cite{GOAT} & - & {\CheckmarkBold} & -\\
NodeFormer~\cite{NodeFormer} &  - & {\CheckmarkBold} & -\\
SGFormer~\cite{SGFormer} &  - & {\CheckmarkBold} & -\\
Polynormer~\cite{Polynormer} &  - & {\CheckmarkBold} & -\\
TunedGNN~\cite{tunedGNN} &  {\CheckmarkBold}* & {\CheckmarkBold} & -\\
\midrule

LSP~\cite{LSP} & - & - & {\CheckmarkBold}\\
GNN-SD~\cite{GNNSD} & - & - & -\\
TinyGNN~\cite{TinyGNN} & - & - & {\CheckmarkBold}\\
GLNN~\cite{GLNN} & - & - & {\CheckmarkBold}\\
KRD~\cite{KRD} & - & - & {\CheckmarkBold}\\

\midrule
\textbf{iPaD(Ours)} & {\CheckmarkBold} & {\CheckmarkBold} & {\CheckmarkBold}\\

\midrule
\end{tabular}
\caption{\textbf{Holistic comparison of representative methods across the core challenges of Completeness, Training Efficiency, and Scalability.} Our proposed {\framework} is the only framework designed to excel in all three aspects.}
\label{tab:holistic_comparison}
\endgroup
\end{table}

% ------------------------------------------------
\EndChapter
% ------------------------------------------------
