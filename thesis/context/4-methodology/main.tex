\StartChapter{Methodology}{chapter:methodology}

This chapter details the architecture and workflow of our proposed \textbf{{\framework}} framework.
As illustrated in Figure~\ref{fig:framework_overview}, {\framework} is an end-to-end pipeline designed to address the challenges of incomplete data, high training costs, and inefficient inference in a unified and synergistic manner.
The entire framework can be understood as a four-stage process: \\
(1) Robust Data Imputation \\
(2) Parallel Teacher-Student Preparation\\
(3) Teacher Training\\
(4) Architecture-Aligned Reliable Knowledge Distillation. \\
The subsequent sections will delve into the technical details of each core module.

\begin{figure}[htbp]
    \centering
    \includegraphics[width=\textwidth]{.//context//fig/prism.png}
    \caption{\textbf{The overall architecture of the {\framework} framework.} The process begins with imputing the incomplete feature matrix via \textbf{APCFI}. In parallel, a powerful GNN teacher is trained on the recovered data, while the \textbf{MPP} module generates a pruned, lightweight MLP student. Finally, the \textbf{MP-KRD} module distills knowledge from the teacher to the student, producing a final model that is both robust and efficient for inference.}
    \label{fig:framework_overview}
\end{figure}

\section{Framework Overview}
\label{sec:overview}

The workflow of our proposed {\framework} framework, as depicted in Figure~\ref{fig:framework_overview}, is a systematic pipeline that transforms a graph with incomplete features into a lightweight, robust, and fast inference model. The process unfolds across three distinct yet interconnected stages:

\begin{enumerate}
    \item \textbf{Stage 1: Robust Data Imputation with APCFI.}
    The process begins with an input graph whose feature matrix, $\mathbf{X}$, contains missing values. To ensure a high-quality foundation for subsequent learning, the graph is first passed through our \textbf{APCFI} (Approximate Pseudo-Confidence Feature Imputation) module.
    APCFI leverages graph diffusion and an \textbf{ASDE} to generate pseudo-confidence scores, which guide a feature propagation process.
    The output of this stage is a fully recovered feature matrix, $\hat{\mathbf{X}}$, where missing entries have been robustly imputed.

    \item \textbf{Stage 2: Synergistic Teacher and Student Generation via MPP.}
    The second stage efficiently prepares both the teacher and student models in a sequential, highly efficient process driven by our \textbf{MPP} module. The workflow is as follows:
    \begin{enumerate}
        \item \textbf{Proxy Pre-training and Pruning:} \\
        We begin with our powerful GNN architecture (e.g., \textbf{TunedGNN}) and project it to its isomorphic MP-MLP. This proxy MLP is then pre-trained and pruned on the target dataset.
        This efficient step determines the optimal sparse structure and directly yields our lightweight \textbf{Student Model} (the ``Pruned MP-MLP").
        \item \textbf{Teacher Architecture Definition:} \\
        The pruned structure from the student model is then inversely projected back to the original GNN architecture. This creates the final, optimized architecture for our teacher: the ``Pruned GNN".

    \end{enumerate}
    This synergistic process is a core innovation of our framework: the lightweight student's architecture is defined in the same efficient step that determines the optimized architecture for the powerful teacher, all before the main, computationally expensive GNN training commences.
    \item \textbf{Stage 3: Teacher Training.} \\ The ``Pruned GNN'' architecture, defined in the previous stage, is fully trained on the imputed dataset $(\hat{\mathbf{X}}, \mathcal{G})$ produced by the APCFI module. This dedicated training phase enables the pruned model to achieve strong predictive performance while maintaining high computational efficiency. The resulting model, referred to as the \textbf{Teacher Model}, serves as a robust and lightweight source of knowledge for the subsequent distillation process. By leveraging both the optimized sparse architecture and the high-quality imputed features, the teacher model is well-suited to guide the training of the student model in the final stage.
    \newpage
    \item \textbf{Stage 4: Knowledge Distillation with MP-KRD.}
    In the final stage, knowledge transfer is orchestrated.
    The \textbf{MP-KRD} module takes the high-performance ``Pruned GNN'' Teacher and the lightweight ``Pruned MP-MLP'' Student.
    It employs a reliability-aware sampling strategy to distill the rich, relational knowledge from the teacher into the student.
    The final output of the entire {\framework} pipeline is this pruned, distilled MLP student—a model that is highly efficient, robust against incomplete data, and ready for fast deployment in real-world applications.
\end{enumerate}

\section{APCFI: Approximate Pseudo-Confidence Feature Imputation}
\label{sec:method_apcfi}

As the first stage of the {\framework} pipeline, the foundational challenge of data incompleteness is addressed.
To this end, \textbf{APCFI} is proposed as a robust and scalable feature completion strategy, advancing beyond prior art by introducing a more sophisticated, two-stage normalization process to enhance imputation quality.
Figure~\ref{fig:apcfi_framework} illustrates the overall workflow of the APCFI module.

\begin{figure}[htbp]
    \centering
    \includegraphics[width=0.9\linewidth]{./context/fig/apcfi.png}
    \caption{\textbf{The overall workflow of the APCFI module.} It takes a graph with an incomplete feature matrix as input and leverages ASDE and Dynamic Joint Channel Diffusion (DJCD) to output a robustly recovered feature matrix for downstream tasks.}
    \label{fig:apcfi_framework}
\end{figure}

\subsection{Motivation and Core Idea}
\label{ssec:apcfi_motivation}

Previous high-performance imputation methods like PCFI~\cite{PCFI} successfully utilize channel-wise confidence to guide feature diffusion. However, this approach faces two key limitations: the calculation of exact shortest paths is computationally expensive, and the raw confidence scores may not optimally reflect the relative importance of nodes.

APCFI is designed to overcome both limitations. Its core philosophy is to improve both the efficiency of confidence calculation and the quality of the diffusion weights. This is achieved through three key innovations:
\begin{enumerate}
    \item \textbf{Approximate Shortest Distance Estimation (ASDE):} Replaces PCFI's costly k-hop traversal with a highly parallelizable message-passing scheme to efficiently approximate shortest-path distances.
    \item \textbf{Soft-Pseudo Confidence (Soft-PC):} Introduces a ``softmax'' layer before the standard row-stochastic normalization to transform raw confidence scores into context-aware, relative reliability weights.
    \item \textbf{Dynamic Joint Channel Diffusion:} Replaces PCFI's sequential diffusion with a parallel process, reducing complexity from $O(d \times k)$ to $O(k)$.
\end{enumerate}

\subsection{Technical Workflow}
\label{ssec:apcfi_workflow}
The APCFI workflow consists of three main steps to transform an incomplete feature matrix $\mathbf{X}$ into a recovered matrix $\hat{\mathbf{X}}$.

\subsubsection{Step 1: Approximate Shortest Distance Estimation (ASDE)}

To efficiently quantify the reliability of feature imputation, the raw pseudo-confidence (PC) is computed for each node-feature pair based on the shortest-path distance (SPD) from a missing-feature node to its nearest observed-feature node in the same channel. The ASDE algorithm, illustrated in Figure~\ref{fig:asde_process}, iteratively propagates distance information through the graph, enabling rapid approximation of these SPD values.

Formally, the shortest-path distance for node $v_i$ and feature channel $d$ is defined as:
\begin{equation}
\text{SPD}_{i,S_d} =
\begin{cases}
1, & \text{if } x_{i,d} \text{ is observed} \\
\min \left\{ t \mid \hat{x}_{i,d}^{(t)} \neq 0 \right\}, & \text{if } x_{i,d} \text{ is missing}
\end{cases}
\end{equation}

The raw pseudo-confidence is then given by:
\begin{equation}
    pc_i^{(d)} = \alpha^{\text{SPD}_{i,S_d}}
    \label{eq:pseudo_confidence}
\end{equation}

where $\alpha \in (0,1)$ is a decay factor that controls the influence of distance on confidence.

\begin{figure}[htbp]
    \centering
    \includegraphics[width=\textwidth]{./context/fig/asde.png}
    \caption{\textbf{Illustration of the Approximate Shortest Distance Estimation (ASDE) process.} The algorithm begins with observed features (step-0) and propagates shortest-path information through the graph. Each entry in the ASDE matrix reflects the shortest hop-distance to an observed feature within the same channel.}
    \label{fig:asde_process}
\end{figure}


\newpage
\subsubsection{Step 2: Two-Stage Normalization}
This step is the core of our imputation mechanism, involving a two-stage normalization process to generate the final diffusion weights.


\subsubsubsection{\textbf{Stage 1: Soft-Pseudo Confidence}}
\\ The raw pseudo-confidence scores $pc_i^{(d)}$ are first passed through a ``softmax'' operation to produce the ``soft confidence matrix''. The goal is to generate context-aware, relative reliability scores, which is a key innovation over PCFI. The resulting weights $w_{i,d}$ are calculated as:
\begin{equation}
    w_{i,d} = \frac{\exp(pc_i^{(d)})}{\sum_{j=1}^{N} \exp(pc_j^{(d)})}
    \label{eq:soft_pc}
\end{equation}

To illustrate the effectiveness of this mechanism, consider the following example. Given a raw pseudo-confidence matrix $\mathbf{PC}$:
\[
\mathbf{PC} =
\begin{bmatrix}
0.5 & 0.5 \\
1.0 & 0.7 \\
0.25 & 1.0
\end{bmatrix}
\]
The resulting soft-pseudo confidence weights $\mathbf{soft\_PC}$ are:
\[
\mathbf{soft\_PC} =
\begin{bmatrix}
0.292 & 0.258 \\
0.481 & 0.316 \\
0.227 & 0.426
\end{bmatrix}
\]
Notably, even though the first node has the same pseudo-confidence value ($0.5$) for both feature channels, its final soft-pseudo confidence weights differ ($0.292$ vs. $0.258$). This is because softmax normalization takes into account the entire pseudo-confidence distribution for each channel, allowing identical values to have different relative importance depending on their context. This makes the weighting process more flexible and robust for the subsequent diffusion step.

\subsubsubsection{\textbf{Stage 2: Row-Stochastic Transition}}
\\ To ensure a stable and mathematically sound diffusion process, the ``soft confidence matrix'' (denoted as $\mathbf{W}'$) undergoes a final row-stochastic normalization. This step is crucial for preserving the feature scale during propagation~\cite{PCFI}. It is achieved by normalizing each weight by its corresponding row sum:
\begin{equation}
    \mathbf{W} = \mathbf{D}^{-1}\mathbf{W}'
\end{equation}
where $\mathbf{D}$ is the diagonal degree matrix with $D_{ij} = \sum_{j} W'_{ij}$. The resulting matrix $\mathbf{W}$ is row-stochastic, meaning each row sums to 1.

\subsubsection{Step 3: Dynamic Joint Channel Diffusion (DJCD)}

\begin{figure}[htbp]
    \centering
    \includegraphics[width=0.9\textwidth]{./context/fig/djcd.png}
    \caption{\textbf{The iterative process of Dynamic Joint Channel Diffusion.} In each of the K steps, a diffusion operation updates the missing values based on neighbors, followed by a ``Fixed'' step where the original, observed feature values are reset to prevent oversmoothing and preserve data fidelity.}
    \label{fig:diffusion_process}
\end{figure}

In this step, joint diffusion is performed based on the normalized weights, enabling efficient feature propagation across multiple channels.

After obtaining the normalized, row-stochastic weight matrix $\mathbf{W}$, the final step involves performing the actual feature diffusion. This iterative process, conducted for a total of $K$ steps, propagates information from observed nodes to missing nodes across the graph.

The update rule for each node $v_i$ and feature channel $d$ at iteration $t+1$ is defined as:
\begin{equation}
    \hat{x}_{i,d}^{(t+1)} =
    \begin{cases}
    x_{i,d}, & \text{if } (i,d) \text{ is observed}\\
    \frac{\sum_{j\in \mathcal{N}(i)} \bar{w}_{ij} \cdot \hat{x}_{j,d}^{(t)}}{\sum_{j\in \mathcal{N}(i)} \bar{w}_{ij}}, & \text{otherwise}
    \end{cases}
    \label{eq:diffusion_update_final}
\end{equation}


For any feature that is already \textbf{observed}, its value is kept fixed.
This is a crucial step to prevent the loss of ground-truth information and mitigate the ``over-smoothing'' problem, a technique also validated in the Feature Propagation (FP)~\cite{FP} paper.


For any \textbf{missing} feature, its value is updated by computing a \textbf{weighted average} of the features of its neighboring nodes from the previous iteration.
The weights $\bar{w}_{ij}$ are precisely the ones we derived from our two-stage normalization process, ensuring that more reliable neighbors contribute more significantly to the imputation.


Notably, due to this design, the diffusion process can be conducted jointly and in parallel across all feature channels, which constitutes a key efficiency advantage of APCFI over PCFI.
The entire iterative process is illustrated in Figure~\ref{fig:diffusion_process}.

\subsubsection{Step 4: Node-wise Inter-Channel Propagation}
As a final refinement, following the primary feature diffusion in the DJCD step, a node-wise inter-channel propagation mechanism is introduced. This step is designed to leverage latent correlations between different feature channels, further fine-tuning the imputed values for each node. The core design is guided by the \textbf{Humility Principle}: nodes with lower confidence in a certain feature value should be more receptive to "suggestions" from other highly correlated features.

The process consists of the following computational steps. Let $\mathbf{\hat{X}}_{\text{djcd}}$ denote the feature matrix recovered from the DJCD step, and let $\mathbf{PC} \in \mathbb{R}^{N \times F}$ be the matrix of pseudo-confidence scores, where each entry $pc_{i,d}$ represents the confidence in the $d$-th feature of node $v_i$.

\begin{enumerate}
    \item \textbf{Inter-Channel Correlation:} The feature correlation matrix $\mathbf{C} \in \mathbb{R}^{F \times F}$ is computed to capture the global linear relationships between all feature channels across the graph:
    $$
    \mathbf{C} = \text{Corrcoef}(\mathbf{\widetilde{X}}_{\text{djcd}}^T)
    $$

    \item \textbf{Confidence-Weighted Signal:} For each node, an initial "suggestion" signal is generated by taking the deviation of each feature from its global mean, weighted by its pseudo-confidence score.
    This ensures that more reliable features contribute more strongly to the suggestion signal. Let $\mathbf{\widetilde{\overline{X}}}_{\text{djcd}}$ be the feature mean vector:
    $$
    \mathbf{M}_{\alpha} = \mathbf{PC} \odot (\mathbf{\widetilde{X}}_{\text{djcd}} - \mathbf{\widetilde{\overline{X}}}_{\text{djcd}})
    $$
    where $\odot$ denotes element-wise multiplication.

    \item \textbf{Suggestion Aggregation:} The signal $\mathbf{M}_{\alpha}$ is propagated through the correlation matrix $\mathbf{C}$, aggregating weighted suggestions from all other channels and enabling features to "advise" each other based on global correlation:
    $$
    \mathbf{M}_{\text{conf}} = \mathbf{M}_{\alpha} \cdot \mathbf{C}
    $$

    \item \textbf{Humility-based Update:} The update term $\Delta\mathbf{X}$ is computed by weighting the aggregated suggestions $\mathbf{M}_{\text{conf}}$ with a "humility" factor $(1 - \mathbf{PC})$ and a hyperparameter $\beta$. The humility factor ensures that features with low initial confidence are updated more significantly:
    $$
    \Delta\mathbf{X} = \beta \cdot (1 - \mathbf{PC}) \odot \mathbf{M}_{\text{conf}}
    $$
    The final imputed feature matrix is obtained by applying this update to the result of the DJCD step:
    $$
    \mathbf{\hat{X}} = \mathbf{\widetilde{X}}_{\text{djcd}} + \Delta\mathbf{X}
    $$
\end{enumerate}

This two-tiered process, combining broad graph-based diffusion with a fine-grained, correlation-aware adjustment, enables APCFI to produce a more robust and nuanced feature representation, providing a solid foundation for downstream GNN models.

\subsection{Implementation and Pseudocode}
\label{ssec:apcfi_pseudo}
The complete APCFI workflow is detailed in the algorithms below (Algorithms~\ref{algo:apcfi} to~\ref{algo:apcfi-node_propagation}).

In summary, by introducing a sophisticated two-stage normalization process and a highly efficient joint diffusion mechanism, APCFI provides a robust and scalable solution for data incomplete.
The recovered feature matrix, $\hat{\mathbf{X}}$, serves as a high-quality data foundation for subsequent model training.
Specifically, the enhanced completeness and reliability of $\hat{\mathbf{X}}$ enable the downstream GNN backbone to better exploit graph structure and node attributes, thereby improving overall learning stability and predictive performance, even in challenging real-world scenarios with severe missingness.
The next section introduces the robust GNN backbone, which is designed to fully leverage this improved feature representation.

\definecolor{green1}{RGB}{40,180,90}
\newcommand{\algstep}[1]{\Statex\State{\textcolor{green1}{\textbf{#1}}}}

\begin{algorithm}[htbp]
    \caption{PyTorch-style Pseudo-code of APCFI}
    \label{algo:apcfi}
    \begin{algorithmic}[1]
        \Function{apcfi\_fill}{feature, edge\_index}
            \State mask = \texttt{~feature.isnan()} % 建議把 ~ 放 code block
            \algstep{Step 1: ASDE and Dynamic Joint Channel Diffusion}
            \State diffusion\_x, asde\_matrix = diffusion(feature, edge\_index, mask)
            \algstep{Step 2: Node-wise Inter-Channel Propagation}
            \State output = node\_propagation(diffusion\_x, asde\_matrix)
            \algstep{Return the imputed feature matrix}
            \State \Return output
        \EndFunction
    \end{algorithmic}
\end{algorithm}

\begin{algorithm}[htbp]
    \caption{PyTorch-style Pseudo-code of Function \texttt{diffusion}}
    \label{algo:apcfi-diffusion}
    \begin{algorithmic}[1]
        \Function{diffusion}{feature, edge\_index, mask}
            \algstep{Step 1: Calculate ASDE matrix}
            \State asde\_matrix = ASDE(edge\_index, mask)

            \algstep{Step 2: Compute pseudo-confidence}
            \State confidence\_matrix = pseudo\_confidence(edge\_index, asde\_matrix)
            \State soft\_confidence\_matrix = F.softmax(confidence\_matrix, dim=1)

            \algstep{Step 3: Normalize by row transition}
            \State normalize\_confidence = row\_transition(soft\_confidence\_matrix, edge\_index[0])

            \algstep{Step 4: Perform channel-wise diffusion}
            \State diffusion\_x = approximate\_channel\_diffusion(feature, normalize\_confidence,
            \Statex \hspace{8.4cm} edge\_index, mask)

            \algstep{Return: Channel-Diffusion(imputed) feature matrix}
            \State \Return diffusion\_x, asde\_matrix
        \EndFunction
    \end{algorithmic}
\end{algorithm}

\begin{algorithm}[htbp]
    \caption{PyTorch-style Pseudo-code of Function \texttt{row\_transition}}
    \label{algo:apcfi-row_transition}
    \begin{algorithmic}[1]
        \Function{row\_transition}{soft\_confidence\_matrix, edge\_index[0]}
            \algstep{Step 1: row-wise sum}
            \State deg\_w = torch\_scatter.scatter\_add(confidence\_matrix, row\_index, dim=0)
            \algstep{Step 2: Inverse degree normalization}
            \State deg\_w\_inv = deg\_w.pow(-1.0)
            \algstep{Step 3: Fill diagonal with zeros}
            \State deg\_w\_inv.masked\_fill_(deg\_w\_inv == float(inf), 0)
            \algstep{Step 4: Normalize confidence matrix}
            \State norm\_confidence\_matrix = confidence\_matrix * deg\_w\_inv[row\_index]
            \algstep{Return: Normalized confidence matrix}
            \State \Return norm\_confidence\_matrix
        \EndFunction
    \end{algorithmic}
\end{algorithm}

\begin{algorithm}[htbp]
    \caption{PyTorch-style Pseudo-code of Function \texttt{node\_propagation}}
    \label{algo:apcfi-node_propagation}
    \begin{algorithmic}[1]
        \Function{node\_propagation}{diffusion\_x, asde\_matrix}

            \algstep{Step 1: Compute inter-channel correlation (conf)}
            \State conf = torch.corrcoef(diffusion\_x.T).nan\_to\_num().fill\_diagonal\_(0)

            \algstep{Step 2: Compute channel mean difference}
            \State alpha\_mean = (alpha ** asde\_matrix) * (diffusion\_x - torch.mean(diffusion\_x, dim=0))

            \algstep{Step 3: Aggregate with channel correlation}
            \State alpha\_mean\_conf = torch.matmul(alpha\_mean, conf)

            \algstep{Step 4: Update node features}
            \State update\_x = beta * (1 - (alpha ** asde\_matrix)) * alpha\_mean\_conf
            \State updated\_x = diffusion\_x + update\_x

            \algstep{Return: Node-updated feature matrix}
            \State \Return updated\_x
        \EndFunction
    \end{algorithmic}
\end{algorithm}

\endgroup

\newpage
\section{TunedGNN: Architecture Optimization for Robustness}
\label{sec:tunedgnn}

Having established a method to obtain a complete and robust feature matrix, the next logical step is to define a GNN architecture capable of effectively learning from this data.
While standard GNNs like GCN or GraphSAGE are widely used, they are often architecturally shallow and can be sensitive to the subtle noise and imperfections that may remain even in imputed data.
To build a powerful and reliable ``teacher" model for our framework, a more resilient foundational architecture is required.



To this end, the \textbf{TunedGNN} is introduced as the backbone model.
It is important to note that \textbf{TunedGNN} is not a novel architecture proposed in this thesis.
Rather, it is a systematic application of several well-established, state-of-the-art architectural best practices from recent GNN research~\cite{tunedGNN}.
The goal is to construct a model with enhanced stability, expressiveness, and, most critically, robustness against data perturbations.
As illustrated in Figure~\ref{fig:tunedgnn_comparison}, the \textbf{TunedGNN} structure is significantly deeper and more complex than classic GNN layers.


\subsection{Key Design Components}
\begin{figure}[htbp]
    \centering
    \begin{subfigure}[b]{.45\linewidth}
        \centering
        \includegraphics[height=0.6\linewidth]{.//context//fig/untrunedGNN.png}
        \caption{\textbf{Classic GNN layer structure}}
        \label{fig:untrunedGNN}
    \end{subfigure}
    \hfill
    \begin{subfigure}[b]{.45\linewidth}
        \centering
        \includegraphics[height=0.9\linewidth]{.//context//fig/trunedGNN.png}
        \caption{\textbf{Tuned GNN layer structure}}
        \label{fig:tunedgnn_comparison}
    \end{subfigure}
    \caption{\textbf{Comparison of Tuned and Classic GNN layer structures.} (a) The classic GNN layer stacks message-passing and activation modules, typically with fewer than 3 layers. (b) The tuned GNN layer incorporates residual connections, normalization, activation, and dropout, allowing for deeper architectures (up to 11 layers) and improved robustness to incomplete data.}
    \label{fig:gnn-comparison}
\end{figure}

The key design components systematically integrated into our \textbf{TunedGNN} backbone are as follows:
\newpage
\begin{itemize}
    \item \textbf{Deeper Layer Stacking:} \\A deeper architecture (e.g., up to 11 layers in the experiments) is employed compared to traditional shallow GNNs. This allows for a larger receptive field, enabling the model to capture more complex, higher-order relational patterns within the graph, while carefully managing the risk of over-smoothing through the components below.

    \item \textbf{Normalization (e.g., LayerNorm):} \\ Each GNN layer is followed by a normalization layer, typically LayerNorm, to stabilize the distribution of intermediate feature representations. This practice is crucial for promoting faster, more stable convergence and is particularly effective in mitigating the impact of noise in high-missingness scenarios.

    \item \textbf{Residual Connections:} \\ To enable effective training of such a deep architecture, skip (or residual) connections are added between layers.
    As shown in Figure~\ref{fig:tunedgnn_comparison}, the output of the message-passing block is added to the original input before subsequent transformations.
    This mitigates the vanishing gradient problem and ensures a smoother gradient flow throughout the network.

    \item \textbf{Dropout:} \\ Dropout layers are applied after each non-linearity or normalization layer.
    This acts as a powerful regularization technique, enhancing the model's generalization capabilities and improving its robustness to perturbations in both node and edge features.
\end{itemize}



The overall layer structure for \textbf{TunedGNN} can be formally summarized by the following sequence of operations for each layer, transforming the input representation $h_i$ to the output $h_{i+1}$:
\begin{align}
    h'_{i} &= \text{Aggregate}(h_{i}, \{h_{j} \mid j \in \mathcal{N}(i)\}) \label{eq:tuned_agg} \\
    h''_{i} &= h'_{i} + \text{Linear}(h_{i}) \label{eq:tuned_res} \\
    h'''_{i} &= \text{Norm}(\sigma(h''_{i})) \label{eq:tuned_norm} \\
    h_{i+1} &= \text{Dropout}(h'''_{i}) \label{eq:tuned_dropout}
\end{align}

where Equation~\ref{eq:tuned_agg} represents the core message-passing step, and Equation~\ref{eq:tuned_res} represents the residual connection.

In conclusion, the \textbf{TunedGNN} architecture provides a highly performant and robust foundation for our framework. However, this power and resilience come at the cost of a large parameter count and significant computational complexity. This directly motivates the need for the next module in our pipeline: an effective and efficient pruning strategy to make this powerful model practical for real-world training and deployment.


\section{MPP: Mirror Projection Pruning}
\label{sec:method_mpp}

The TunedGNN~\cite{tunedGNN} architecture, as described in the previous section, provides a powerful and robust foundation for our framework. However, its depth and complexity lead to high computational costs during training. To address this challenge, we introduce \textbf{MPP}, a novel paradigm designed to efficiently reduce model complexity without sacrificing representational capacity.

Existing GNN pruning techniques are often tightly intertwined with the training process, resulting in significant computational overhead and a lack of flexibility.
In contrast, MPP introduces a \textbf{proxy-based, pre-training pruning} strategy.
Its core idea is to decouple the complex pruning task by projecting the GNN to an isomorphic MLP, applying mature pruning techniques in the simpler MLP space, and then mapping the pruned structure back. This design, inspired by studies like MLPinit~\cite{MLPinit} that demonstrate GNN-MLP parameter alignment, allows us to leverage a rich ecosystem of existing pruning tools, such as DepGraph (torch-pruning)~\cite{fang2023depgraph}. Crucially, MPP serves a dual purpose in our pipeline: (1) it produces an efficient, pruned GNN to act as our \textbf{Teacher model}, and (2) it simultaneously generates a lightweight, structurally-aligned MLP to serve as our \textbf{Student model} for the final distillation stage.

\subsection{Mirror Projection Pruning (MPP): Theory and Workflow}
\label{ssec:mpp_workflow_final}
Mirror Projection Pruning (MPP) (See Figure~\ref{fig:mpp_workflow}) is built on the structural isomorphism between common GNN layers and their MLP counterparts, extending the insight of MLPInit~\cite{MLPinit} that an MLP with the same architecture as a GNN can serve as a parameter proxy.
\begin{figure}[H]
    \centering
    \includegraphics[width=0.6\textwidth]{./context/fig/mpp.png}
    \caption{\textbf{The high-level workflow of Mirror Projection Pruning (MPP).} A large GNN architecture is projected to a proxy MP-MLP, which is then trained and pruned. The resulting sparse structure is projected back to define the final, efficient `Pruned GNN'.}
    \label{fig:mpp_workflow}
\end{figure}

\newpage
Our approach generalizes this concept: an isomorphic MLP can act as an efficient surrogate for a GNN, enabling all training, pruning, and architectural search to be performed in the lightweight MLP domain. Afterward, the resulting sparse structure and weights are seamlessly mapped back to the original GNN, yielding an efficient, pruned GNN with minimal engineering effort.

\subsubsection{The Principle of GNN–MLP Isomorphism}

The isomorphism between GNN and MLP layers is achieved by removing the neighborhood aggregation (message-passing) operation from the GNN, resulting in a pure node-level transformation (See Figure~\ref{fig:mp-mlp_view}).

\begin{figure}[ht]
    \centering
    \includegraphics[width=0.68\textwidth]{./context/fig/mp-mlp_view.png}
    \caption{
    \textbf{Structural illustration of SAGEConv and its MP-MLP projection.}
    Left: The SAGEConv layer consists of two linear branches, one operating on neighborhood-aggregated features (Propagate).
    Right: The SAGEConvMLP projection replaces neighborhood aggregation with a second self-branch, yielding a pure local MLP structure with identical parameterization.
    This one-to-one mapping is fundamental to efficient pruning and parameter transfer in MPP.
    }
    \label{fig:mp-mlp_view}
\end{figure}
For example, a SAGEConv layer comprises two linear branches—one operating on the node's own features, the other on the aggregated features from its neighbors. By eliminating the propagation step, we obtain an MLP layer (SAGEConvMLP) with two self-branches, preserving both parameterization and output dimension.

This theoretical foundation ensures that all layers, hidden dimensions, and normalization structures in the original GNN are exactly matched by the MP-MLP, making direct parameter and mask transfer possible in both directions.

\newpage
\subsubsection{Step 1: Mirror Projection---GNN to MP-MLP}

Given a TunedGNN backbone (e.g., TunedGNN~\cite{tunedGNN}), the first step is to construct a mirror-projected MLP (MP-MLP) by replacing each GNN-layer with its isomorphic ConvMLP layer. All message-passing operations are removed, while linear transformations, normalization, and nonlinearity are preserved. This results in an untrained MP-MLP network that is structurally identical to the original GNN, as shown in Figure~\ref{fig:mpp_step1}.

\begin{figure}[ht]
    \centering
    \includegraphics[width=0.92\textwidth]{./context/fig/mpp_step1.png}
    \caption{\textbf{Mirror Projection from GNN to MP-MLP.}
    Each GNN-layer is replaced by a corresponding ConvMLP layer, yielding an untrained MP-MLP that mirrors the GNN architecture.}
    \label{fig:mpp_step1}
\end{figure}

\subsubsection{Step 2: Training and Pruning in MP-MLP Space}

All subsequent training and pruning are performed in the MP-MLP domain. The MP-MLP is first trained on the completed feature set for the target task, then pruned using state-of-the-art MLP pruning algorithms (e.g., LAMP~\cite{lee2020layer}, DepGraph~\cite{fang2023depgraph}), which efficiently remove redundant weights or channels based on importance. This step is illustrated in Figure~\ref{fig:mpp_step2}.

\begin{figure}[ht]
    \centering
    \includegraphics[width=0.92\textwidth]{./context/fig/mpp_step2.png}
    \caption{\textbf{Proxy Training and Pruning in MP-MLP.}
    The MP-MLP is trained and pruned to yield a sparse, efficient student model.}
    \label{fig:mpp_step2}
\end{figure}

\subsubsection{Step 3: Inverse Projection---Parameter and Mask Transfer Back to GNN}

Once pruning is completed, the sparse weights and binary pruning mask from the MP-MLP are mapped directly back to the original GNN, thanks to the strict structural alignment. Each pruned ConvMLP layer's parameters and mask are copied to the corresponding GNN-layer, transforming the large, dense GNN into a compact, sparse network without modifying the original codebase. See Figure~\ref{fig:mpp_step3}.

\begin{figure}[ht]
    \centering
    \includegraphics[width=0.92\textwidth]{./context/fig/mpp_step3.png}
    \caption{\textbf{Inverse Projection from Pruned MP-MLP to Pruned GNN.}
    The pruning mask and weights are transferred directly, instantly producing a sparse, efficient GNN.}
    \label{fig:mpp_step3}
\end{figure}

\subsubsection{Advantages of MPP}

\begin{itemize}
    \item \textbf{Computational efficiency:} Training and pruning are much faster in the MLP domain, as no neighborhood aggregation is needed.
    \item \textbf{Structural flexibility:} Any MLP pruning or search technique can be applied to the proxy MP-MLP and directly benefit the GNN.
    \item \textbf{Seamless transfer:} Strict one-to-one mapping enables direct synchronization of parameters and masks between MLP and GNN.
    \item \textbf{Minimal engineering overhead:} No modification to the GNN implementation is required; all changes are applied through parameter transfer.
\end{itemize}

This mirror projection--pruning--inverse projection workflow offers a highly flexible, efficient, and practical GNN compression paradigm, making advanced pruning and architectural optimization accessible for real-world graph learning tasks.

\newpage
\subsubsection{Implementation and Pseudocode}
The entire process is outlined in Algorithm~\ref{algo:mpp}.
This workflow efficiently produces a pruned teacher GNN and a structurally-aligned student MLP, perfectly setting the stage for our final knowledge transfer module.

\definecolor{green1}{RGB}{40,180,90}
\newcommand{\algstep}[1]{\Statex\State{\textcolor{green1}{\textbf{#1}}}}

\begin{algorithm}[H]
    \caption{MPP Proxy Pruning Pseudocode}
    \label{algo:mpp}
    \begin{algorithmic}[1]
        \Function{mpp\_prune}{gnn\_model, feature, label}
        \algstep{Step 1: Remove message-passing and project to MLP proxy}
        \State \texttt{mp\_mlp = project\_to\_mlp(gnn\_model)}
        \algstep{Step 2: Pre-train the proxy model}
        \State \texttt{mp\_mlp.train(feature, label)}
        \algstep{Step 3: Prune the proxy MLP}
        \State \texttt{pruning\_mask = prune\_mlp(mp\_mlp, method="LAMP")}
        \algstep{Step 4: Generate the student model}
        \State \texttt{pruned\_mlp = apply\_mask(mp\_mlp, pruning\_mask)}
        \algstep{Step 5: Parameter and Mask Transfer back to original GNN (teacher model)}
        \State \texttt{pruned\_gnn = apply\_mask(gnn\_model, pruning\_mask)}
        \algstep{Return the pruned GNN and pruned MLP}
        \State \Return \texttt{pruned\_gnn, pruned\_mlp}
        \EndFunction
    \end{algorithmic}
\end{algorithm}
\endgroup





\section{Teacher Training}
\label{sec:training}

After obtaining the pruned GNN from the MPP workflow , a dedicated training phase is conducted to optimize its parameters on the imputed dataset generated by the APCFI module. Specifically, the pruned GNN is trained using the recovered feature matrix $\widehat{\mathbf{X}}$ and the original graph structure, following standard supervised learning protocols for node classification.

Throughout this stage (see Fig.~\ref{fig:teacher_training}), the model benefits from both the lightweight architecture achieved via pruning and the improved data quality resulting from feature imputation. The cross-entropy loss is adopted as the objective function to maximize classification accuracy. Upon completion of training, the resulting high-performance and efficient GNN is designated as the \textbf{Teacher Model} for the subsequent knowledge distillation phase.

This process ensures that the teacher model not only possesses strong predictive capabilities but also maintains computational efficiency, thereby serving as an ideal source of knowledge for transferring to a student model.

\begin{figure}[h]
    \centering
    \includegraphics[width=0.7\linewidth]{./context/fig/training.png} % 根據實際圖片檔名修改
    \caption{
        \textbf{Teacher Training Workflow.}
        The pruned GNN obtained from the MPP module is fully trained on the imputed feature matrix $\widehat{\mathbf{X}}$ generated by APCFI. The resulting efficient model is used as the Teacher for knowledge distillation.
    }
    \label{fig:teacher_training}
\end{figure}

\section{MP-KRD: Mirror Projection Knowledge-inspired Reliable Distillation}
\label{sec:method_mpkrd}

The final stage of our {\framework} pipeline, \textbf{MP-KRD}, orchestrates the knowledge transfer from the powerful teacher model to the lightweight student model, both of which were prepared by the preceding MPP module.
Specifically, it takes the trained ``Pruned TunedGNN'' as the Teacher and the ``Pruned MP-MLP'' as the Student. As depicted in Figure~\ref{fig:mpkrd_high_level}, the goal is to create a final inference model that is both highly efficient and robust.

\begin{figure}[ht]
    \centering
    \includegraphics[width=.6\textwidth]{./context/fig/mp-rkd.png}
    \caption{\textbf{The high-level view of the MP-KRD module.} It takes the trained GNN Teacher and the Pruned MP-MLP Student as input. A Reliable Label Sampler, guided by a curriculum, selects high-quality knowledge to distill, producing the final, efficient inference model.}
    \label{fig:mpkrd_high_level}
\end{figure}

\newpage
\subsection{Motivation: The Need for Reliable Distillation}
\label{ssec:mpkrd_motivation}

A naive distillation process that simply forces the student to mimic all of the teacher's outputs can be problematic.
A teacher trained on imputed or noisy data may exhibit low confidence or even make mistakes on certain samples.
Forcing the student to learn this ``unreliable knowledge'' can harm its performance.
To motivate our approach, consider two samples (A and B) with the same ground-truth label $[0,1,0]$, where the teacher GNN produces the following output logits:

\begin{itemize}
    \item \textbf{Sample A:} GNN output logits $[0.3,\, 0.5,\, 0.2]$
    \item \textbf{Sample B:} GNN output logits $[0.075,\, 0.9,\, 0.025]$
\end{itemize}

Although both samples are correctly classified, the cross-entropy loss for Sample B is noticeably lower than for Sample A.
Intuitively, this means the model is more confident in its prediction for Sample B.
Prediction reliability is systematically quantified using a robust mechanism rather than relying solely on raw logit values.

Therefore, instead of treating all teacher knowledge equally, MP-KRD employs a \textbf{reliability-aware curriculum}.
The core idea is to quantify sample ``difficulty" or ``unreliability" by the \textbf{stability of the teacher's prediction under noise perturbation},
measured via entropy change. This allows the student to learn from the most stable and reliable knowledge first.

\subsection{The MP-KRD Workflow}
\label{ssec:mpkrd_workflow}
The MP-KRD process is an iterative workflow that adaptively transfers knowledge from the teacher to the student (see Fig.~\ref{fig:mpkrd_workflow}).

\begin{figure}[ht]
    \centering
    \includegraphics[width=.6\textwidth]{./context/fig/mp-rkd_training.png}
    \caption{\textbf{The workflow of the MP-KRD module.} The pruned GNN teacher provides sample reliability signals for curriculum-based distillation, guiding the training of the MP-MLP student via cross-entropy, logit KD (KRD), and feature KD (CWD) losses.}
    \label{fig:mpkrd_workflow}
\end{figure}

\newpage
\subsubsection{\textbf{Step 1: Sample Difficulty Calculation.}}
The workflow begins by quantifying the reliability of the teacher's prediction for each training sample $i$.
This is done by measuring the change in the teacher's output entropy when a small Gaussian noise, $\epsilon \sim \mathcal{N}(0, \delta)$, is added to the input features $\mathbf{X}$.

\begin{align}
    \Delta H_i &= | H(f_{\text{GNN}}(\mathbf{X}', i)) - H(f_{\text{GNN}}(\mathbf{X}, i)) | \\
    \Delta e_i &= \frac{\Delta H_i - \min_j \Delta H_j}{\max_j \Delta H_j - \min_j \Delta H_j} \label{eq:norm_entropy}
\end{align}
The normalized entropy change, $\Delta e_i \in [0, 1]$, serves as our metric for sample difficulty, where a higher value indicates a less reliable or more difficult sample.

\subsubsection{\textbf{Step 2: Curriculum-based Sampling.}}
In each training epoch, we use a curriculum to select a subset of reliable samples.
The probability of selecting sample $i$ is governed by a dynamic power function:
$$ \mathrm{p_i} = 1 - \Delta e_i^{\text{power}} $$
Based on this probability, a mask is generated via Bernoulli sampling, and only the selected samples ($S_{\text{epoch}}$) are used for training in that epoch.
This ensures that easier samples (low $\Delta e_i$) are prioritized in the early stages of training.

\subsubsection{\textbf{Step 3: Composite Distillation Objective.}}
The student model is trained on the sampled subset $S_{\text{epoch}}$ using a composite loss function designed to capture knowledge from multiple perspectives:
\begin{equation}
L_{\text{MP-KRD}} = L_{\mathrm{CE}} + \lambda L_{\mathrm{KRD}} + \alpha L_{\mathrm{CWD}}
\end{equation}
The hyperparameters $\lambda$ and $\alpha$ balance the contributions of these different forms of knowledge.

\begin{itemize}
    \item \textbf{Cross-entropy loss} ($L_{\mathrm{CE}}$): The standard supervised cross-entropy loss against the ground-truth labels,
    \begin{equation}
        L_{\mathrm{CE}} = -\sum_{i \in N} y_{i} \log \left(\hat{y}_i^S \right)
    \end{equation}
    where $y_i$ is the true label and $\hat{y}_i^S$ denotes the student model's softmax output.

    \newpage
    \item \textbf{Logit-based distillation loss} ($L_{\mathrm{KRD}}$): Encourages the student's output distribution to match the teacher's,
    \begin{equation}
        L_{\mathrm{KRD}} = \sum_{i \in S_{\text{epoch}}} \mathrm{KL}(\text{Student}_{\text{logit}}(x_i),\, \text{Teacher}_{\text{logit}}(x_i))
    \end{equation}
    where $\mathrm{KL}$ denotes the Kullback-Leibler divergence, and $S_{\text{epoch}}$ is the set of training samples selected for distillation at the current epoch (e.g., the sampled reliable points under the curriculum policy).

    \item \textbf{Feature-based distillation loss} ($L_{\mathrm{CWD}}$): Pushes the student's hidden representations ($z^{\text{student}}_i$) to align with the teacher's ($z^{\text{teacher}}_i$)\cite{CWD},
    \begin{align}
        L_{\text{node-wise}}^l &= \mathrm{KL}\left( \mathrm{Softmax}(\mathbf{Z}^{S,l}),\; \mathrm{Softmax}(\mathbf{Z}^{T,l}) \right) \\
        L_{\text{channel-wise}}^l &= \mathrm{KL}\left( \mathrm{Softmax}\left((\mathbf{Z}^{S,l})^{\top}\right),\; \mathrm{Softmax}\left((\mathbf{Z}^{T,l})^{\top}\right) \right) \\
        L_{\mathrm{CWD}}^l &= L_{\text{node-wise}}^l + L_{\text{channel-wise}}^l \\
        L_{\mathrm{CWD}}^{\text{total}} &= \sum_{l \in L} L_{\mathrm{CWD}}^l
    \end{align}
    where $l$ indexes the selected layers for distillation, and $L$ is the set of such layers.
\end{itemize}

\subsubsection{\textbf{Step 4: Adaptive Curriculum Update.}}
To make the curriculum dynamic, the `power' parameter is updated at the end of each epoch.
This is done by analyzing the student's current performance against the teacher's.


In order to investigate how sample difficulty affects the consistency between teacher and student predictions, all training samples are partitioned into discrete bins based on their normalized entropy change ($\Delta e$).
For each bin, the frequencies of teacher-student matches and mismatches are computed, and their distributions are illustrated in Figure~\ref{fig:hist_agreement}.
This provides an empirical foundation for the subsequent curriculum adjustment process.

\begin{figure}[h!]
    \centering
    \begin{subfigure}[b]{.475\linewidth}
        \centering
        \includegraphics[width=\linewidth]{./context/fig/hist_true.png}
        \caption{\textbf{Teacher-student agreement histogram}}
        \label{fig:hist_true}
    \end{subfigure}
    \hfill
    \begin{subfigure}[b]{.475\linewidth}
        \centering
        \includegraphics[width=\linewidth]{./context/fig/hist_false.png}
        \caption{\textbf{Teacher-student disagreement histogram}}
        \label{fig:hist_false}
    \end{subfigure}
    \caption{\textbf{Histograms of teacher-student agreement (a) and disagreement (b) plotted against the normalized entropy change.} Easier samples (low entropy change) show high agreement, while harder samples show high disagreement.}
    \label{fig:hist_agreement}
\end{figure}

To quantitatively measure the agreement probability within each entropy bin, the following metric is defined:

\begin{equation}
    hist = \frac{hist_{true}}{hist_{true} + hist_{false} + \varepsilon}
\end{equation}
where $hist_{true}$ and $hist_{false}$ denote the number of agreement and disagreement samples in $\varepsilon$ is a small constant for numerical stability.

Next, as shown in Figure~\ref{fig:power_fitting}, we fit the empirical probabilities {hist$_{fin}$} using the function:
\begin{equation}
    p = 1 - \Delta e^{\text{power}_{fit}}
\end{equation}

\begin{equation}
    \text{power}_{fit} = \frac{\log(1 - p)}{\log(\Delta e)}
\end{equation}

where $\Delta e$ is the normalized entropy change and $\text{power}_{fit}$ is determined by \texbf{least-squares fitting}.
This process dynamically determines the optimal curriculum power parameter for the current epoch.

\begin{figure}[ht]
    \centering
    \includegraphics[width=0.75\linewidth]{./context/fig/hist_fit.png}
    \caption{The empirical agreement probability ($p_k$) is fitted with the power curve (red line) to dynamically determine the optimal curriculum ``power$_{fit}$'' parameter.}
    \label{fig:power_fitting}
\end{figure}

\newpage
Finally, the ``power'' for the next epoch is updated using an \textbf{Exponential Moving Average (EMA)} to ensure a smooth curriculum progression:
\begin{equation}
    \text{power}(t) = \beta \cdot \text{power}(t-1) + (1 - \beta) \cdot \text{power}_{fit}(t)
\end{equation}
where $\beta$ is a smoothing coefficient(momentum) and $\text{power}_{fit}(t)$ is the fitted value from the previous step.

\subsection{Deployment}
\label{sec:mpkrd_deployment}
After MP-KRD training, only the sparse pruned mirror-projected MLP student is deployed, achieving both high efficiency and robustness through distilled GNN knowledge.



\newpage
\section{Innovation Summary}
\label{sec:Innovation_Summary}
Tables~\ref{tab:apcfi_innovations}–\ref{tab:mpkrd_innovations} provide a component-wise classification of the techniques integrated into the framework, categorized into \textbf{adoption}, \textbf{improvement}, or \textbf{original innovation}. 
This classification specifies the methodological sources and identifies the components that constitute original contributions.

Within the \textbf{Imputation Module (APCFI)}, the \textbf{Parallel Diffusion} mechanism is classified as an \textbf{original innovation}, enabling simultaneous multi-channel feature estimation to enhance computational stability and scalability on large-scale graphs.  
In the \textbf{Pruning Module (MPP)}, the \textbf{Proxy Pruning} strategy, also categorized as an \textbf{original innovation}, employs a low-cost MP-MLP proxy during pruning to accelerate both pruning and GNN training, while improving the effectiveness of knowledge distillation.  
For the \textbf{Distillation Module (MP-KRD)}, the \textbf{Homogeneous Student Structure} represents an \textbf{original innovation} that resolves structural mismatches between teacher and student models, and the modified \textbf{CWD Loss} improves the efficiency and stability of channel-wise knowledge transfer.

\begin{landscape}
    \begin{table}[ht]
        \centering
        % 定義底色
\definecolor{lightgray}{gray}{0.95}
\definecolor{blue1}{RGB}{0,128,255}
\definecolor{orange1}{RGB}{235,120,23}
\definecolor{green1}{RGB}{20,170,70}
\definecolor{mygray}{gray}{0.825}


\rowcolors{2}{mygray}{white}


\renewcommand{\arraystretch}{1.1} % 增加行高,表格更寬鬆




    \begin{tabular}{L{0.15\textwidth}|C{0.15\textwidth}L{0.75\textwidth}|C{0.14\textwidth}}
    \textbf{\small Technical Component} & \textbf{\small Source} & \textbf{\small Innovation Description} & \textbf{\small Innovation Category}\\
    \midrule
    \textbf{\small Feature-Propagation} & \cite{FP} & {\small Adopted standard feature propagation as the fundamental step for missing feature estimation} & Adoption\\
    \textbf{\small Pseudo-Confidence} & \cite{PCFI} & {\small Adopted existing pseudo-confidence calculation method as the basis for feature reliability estimation} & Adoption \\
    \textbf{\small Parallel Diffusion} & \textbf{Our} & {\small Designed a parallel diffusion strategy to perform feature estimation simultaneously on multiple subgraph channels, improving computational efficiency and stability for large-scale graphs} & Original Innovation\\

    \bottomrule
    \end{tabular}
    \caption{\textbf{\small Imputation Module (APCFI) Innovations.} Detailed breakdown of the APCFI components, specifying the source of each technique and categorizing the level of innovation.}
    \label{tab:apcfi_innovations}
        \vspace{2\baselineskip}
        % 定義底色
\definecolor{lightgray}{gray}{0.95}
\definecolor{blue1}{RGB}{0,128,255}
\definecolor{orange1}{RGB}{235,120,23}
\definecolor{green1}{RGB}{20,170,70}
\definecolor{mygray}{gray}{0.825}



\rowcolors{2}{mygray}{white}


\renewcommand{\arraystretch}{1.1} % 增加行高,表格更寬鬆




    \begin{tabular}{L{0.15\textwidth}|C{0.15\textwidth}L{0.75\textwidth}|C{0.14\textwidth}}
    \textbf{\small Technical Component} & \textbf{\small Source} & \textbf{\small Innovation Description} & \textbf{\small Innovation Category}\\
    \midrule
    \textbf{\small Parameter Transfer} & \cite{MLPinit} & {\small Redesigned parameter transfer to adapt projection and weight mapping for GNNs, improving stability after pruning} & Adoption\\
    \textbf{\small Channel Pruning} & \cite{fang2023depgraph, lee2020layer} & {\small Adopted conventional channel pruning strategies without modifying the original computation and selection criteria} & Adoption \\
    \textbf{\small Proxy Pruning} & \textbf{Our} & {\small Proposed replacing GNN with a low-cost MLP as a proxy model during pruning. The proxy significantly accelerates pruning and GNN training, while enhancing the effectiveness of knowledge distillation by providing a lightweight yet structure-aware student model} & Original Innovation\\

    \bottomrule
    \end{tabular}
    \caption{\textbf{\small Pruning Module (MPP) Innovations.} Comprehensive summary of pruning techniques, including parameter transfer, channel pruning, and the proposed proxy pruning strategy using MLP to accelerate GNN training and enhance KD.}
    \label{tab:mpp_innovations}

    \end{table}

    \begin{table}[ht]
        \centering
        % 定義底色
\definecolor{lightgray}{gray}{0.95}
\definecolor{blue1}{RGB}{0,128,255}
\definecolor{orange1}{RGB}{235,120,23}
\definecolor{green1}{RGB}{20,170,70}
\definecolor{mygray}{gray}{0.825}



\rowcolors{2}{mygray}{white}


\renewcommand{\arraystretch}{1.1} % 增加行高,表格更寬鬆




    \begin{tabular}{L{0.15\textwidth}|C{0.15\textwidth}L{0.75\textwidth}|C{0.14\textwidth}}
    \textbf{\small Technical Component} & \textbf{\small Source} & \textbf{\small Innovation Description} & \textbf{\small Innovation Category}\\
    \midrule
    \textbf{\small KDMLP} & \cite{GLNN} & {\small Adapted existing MLP-based distillation methods to fit GNN architectures, applied to student models generated by MPP pruning} & Adoption\\
    \textbf{\small Curriculum Learning} & \cite{KRD} & {\small Incorporated curriculum learning strategies into the distillation process, gradually increasing sample difficulty to enhance student model learning capability} & Adoption \\
    \textbf{\small Homogeneous Student Structure} & \textbf{Our} & {\small Proposed mp-MLP-based student model design to align its structure with the teacher’s embedding space, addressing teacher–student structure mismatch in distillation} & Original Innovation\\
    \textbf{\small Channel-wise Distillation Loss} & \cite{CWD} & {\small Modified the existing CWD Loss to adapt weighting and regularization for GNN feature channels, improving the efficiency and stability of knowledge transfer} & Improvement\\

    \bottomrule
    \end{tabular}
    \caption{\textbf{Distillation Module (MP-KRD) Innovations.} Overview of MP-KRD components, covering KDMLP, curriculum learning, the proposed homogeneous student structure, and the improved CWDLoss for GNN adaptation.}
    \label{tab:mpkrd_innovations}
    \end{table}
\end{landscape}



% ------------------------------------------------
\EndChapter
% ------------------------------------------------
