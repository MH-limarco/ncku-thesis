\section{Innovation Summary}
\label{sec:Innovation_Summary}
Tables~\ref{tab:apcfi_innovations}–\ref{tab:mpkrd_innovations} provide a component-wise classification of the techniques integrated into the framework, categorized into \textbf{adoption}, \textbf{improvement}, or \textbf{original innovation}. 
This classification specifies the methodological sources and identifies the components that constitute original contributions.

Within the \textbf{Imputation Module (APCFI)}, the \textbf{Parallel Diffusion} mechanism is classified as an \textbf{original innovation}, enabling simultaneous multi-channel feature estimation to enhance computational stability and scalability on large-scale graphs.  
In the \textbf{Pruning Module (MPP)}, the \textbf{Proxy Pruning} strategy, also categorized as an \textbf{original innovation}, employs a low-cost MP-MLP proxy during pruning to accelerate both pruning and GNN training, while improving the effectiveness of knowledge distillation.  
For the \textbf{Distillation Module (MP-KRD)}, the \textbf{Homogeneous Student Structure} represents an \textbf{original innovation} that resolves structural mismatches between teacher and student models, and the modified \textbf{CWD Loss} improves the efficiency and stability of channel-wise knowledge transfer.

\begin{landscape}
    \begin{table}[ht]
        \centering
        % 定義底色
\definecolor{lightgray}{gray}{0.95}
\definecolor{blue1}{RGB}{0,128,255}
\definecolor{orange1}{RGB}{235,120,23}
\definecolor{green1}{RGB}{20,170,70}
\definecolor{mygray}{gray}{0.825}


\rowcolors{2}{mygray}{white}


\renewcommand{\arraystretch}{1.1} % 增加行高,表格更寬鬆




    \begin{tabular}{L{0.15\textwidth}|C{0.15\textwidth}L{0.75\textwidth}|C{0.14\textwidth}}
    \textbf{\small Technical Component} & \textbf{\small Source} & \textbf{\small Innovation Description} & \textbf{\small Innovation Category}\\
    \midrule
    \textbf{\small Feature-Propagation} & \cite{FP} & {\small Adopted standard feature propagation as the fundamental step for missing feature estimation} & Adoption\\
    \textbf{\small Pseudo-Confidence} & \cite{PCFI} & {\small Adopted existing pseudo-confidence calculation method as the basis for feature reliability estimation} & Adoption \\
    \textbf{\small Parallel Diffusion} & \textbf{Our} & {\small Designed a parallel diffusion strategy to perform feature estimation simultaneously on multiple subgraph channels, improving computational efficiency and stability for large-scale graphs} & Original Innovation\\

    \bottomrule
    \end{tabular}
    \caption{\textbf{\small Imputation Module (APCFI) Innovations.} Detailed breakdown of the APCFI components, specifying the source of each technique and categorizing the level of innovation.}
    \label{tab:apcfi_innovations}
        \vspace{2\baselineskip}
        % 定義底色
\definecolor{lightgray}{gray}{0.95}
\definecolor{blue1}{RGB}{0,128,255}
\definecolor{orange1}{RGB}{235,120,23}
\definecolor{green1}{RGB}{20,170,70}
\definecolor{mygray}{gray}{0.825}



\rowcolors{2}{mygray}{white}


\renewcommand{\arraystretch}{1.1} % 增加行高,表格更寬鬆




    \begin{tabular}{L{0.15\textwidth}|C{0.15\textwidth}L{0.75\textwidth}|C{0.14\textwidth}}
    \textbf{\small Technical Component} & \textbf{\small Source} & \textbf{\small Innovation Description} & \textbf{\small Innovation Category}\\
    \midrule
    \textbf{\small Parameter Transfer} & \cite{MLPinit} & {\small Redesigned parameter transfer to adapt projection and weight mapping for GNNs, improving stability after pruning} & Adoption\\
    \textbf{\small Channel Pruning} & \cite{fang2023depgraph, lee2020layer} & {\small Adopted conventional channel pruning strategies without modifying the original computation and selection criteria} & Adoption \\
    \textbf{\small Proxy Pruning} & \textbf{Our} & {\small Proposed replacing GNN with a low-cost MLP as a proxy model during pruning. The proxy significantly accelerates pruning and GNN training, while enhancing the effectiveness of knowledge distillation by providing a lightweight yet structure-aware student model} & Original Innovation\\

    \bottomrule
    \end{tabular}
    \caption{\textbf{\small Pruning Module (MPP) Innovations.} Comprehensive summary of pruning techniques, including parameter transfer, channel pruning, and the proposed proxy pruning strategy using MLP to accelerate GNN training and enhance KD.}
    \label{tab:mpp_innovations}

    \end{table}

    \begin{table}[ht]
        \centering
        % 定義底色
\definecolor{lightgray}{gray}{0.95}
\definecolor{blue1}{RGB}{0,128,255}
\definecolor{orange1}{RGB}{235,120,23}
\definecolor{green1}{RGB}{20,170,70}
\definecolor{mygray}{gray}{0.825}



\rowcolors{2}{mygray}{white}


\renewcommand{\arraystretch}{1.1} % 增加行高,表格更寬鬆




    \begin{tabular}{L{0.15\textwidth}|C{0.15\textwidth}L{0.75\textwidth}|C{0.14\textwidth}}
    \textbf{\small Technical Component} & \textbf{\small Source} & \textbf{\small Innovation Description} & \textbf{\small Innovation Category}\\
    \midrule
    \textbf{\small KDMLP} & \cite{GLNN} & {\small Adapted existing MLP-based distillation methods to fit GNN architectures, applied to student models generated by MPP pruning} & Adoption\\
    \textbf{\small Curriculum Learning} & \cite{KRD} & {\small Incorporated curriculum learning strategies into the distillation process, gradually increasing sample difficulty to enhance student model learning capability} & Adoption \\
    \textbf{\small Homogeneous Student Structure} & \textbf{Our} & {\small Proposed mp-MLP-based student model design to align its structure with the teacher’s embedding space, addressing teacher–student structure mismatch in distillation} & Original Innovation\\
    \textbf{\small Channel-wise Distillation Loss} & \cite{CWD} & {\small Modified the existing CWD Loss to adapt weighting and regularization for GNN feature channels, improving the efficiency and stability of knowledge transfer} & Improvement\\

    \bottomrule
    \end{tabular}
    \caption{\textbf{Distillation Module (MP-KRD) Innovations.} Overview of MP-KRD components, covering KDMLP, curriculum learning, the proposed homogeneous student structure, and the improved CWDLoss for GNN adaptation.}
    \label{tab:mpkrd_innovations}
    \end{table}
\end{landscape}
