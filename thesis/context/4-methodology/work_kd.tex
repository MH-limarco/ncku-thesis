% 定義底色
\definecolor{lightgray}{gray}{0.95}
\definecolor{blue1}{RGB}{0,128,255}
\definecolor{orange1}{RGB}{235,120,23}
\definecolor{green1}{RGB}{20,170,70}
\definecolor{mygray}{gray}{0.825}



\rowcolors{2}{mygray}{white}


\renewcommand{\arraystretch}{1.1} % 增加行高,表格更寬鬆




    \begin{tabular}{L{0.15\textwidth}|C{0.15\textwidth}L{0.75\textwidth}|C{0.14\textwidth}}
    \textbf{\small Technical Component} & \textbf{\small Source} & \textbf{\small Innovation Description} & \textbf{\small Innovation Category}\\
    \midrule
    \textbf{\small KDMLP} & \cite{GLNN} & {\small Adapted existing MLP-based distillation methods to fit GNN architectures, applied to student models generated by MPP pruning} & Adoption\\
    \textbf{\small Curriculum Learning} & \cite{KRD} & {\small Incorporated curriculum learning strategies into the distillation process, gradually increasing sample difficulty to enhance student model learning capability} & Adoption \\
    \textbf{\small Homogeneous Student Structure} & \textbf{Our} & {\small Proposed mp-MLP-based student model design to align its structure with the teacher’s embedding space, addressing teacher–student structure mismatch in distillation} & Original Innovation\\
    \textbf{\small Channel-wise Distillation Loss} & \cite{CWD} & {\small Modified the existing CWD Loss to adapt weighting and regularization for GNN feature channels, improving the efficiency and stability of knowledge transfer} & Improvement\\

    \bottomrule
    \end{tabular}
    \caption{\textbf{Distillation Module (MP-KRD) Innovations.} Overview of MP-KRD components, covering KDMLP, curriculum learning, the proposed homogeneous student structure, and the improved CWDLoss for GNN adaptation.}
    \label{tab:mpkrd_innovations}