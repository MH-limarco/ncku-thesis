\StartChapter{Complex Missing Conditions}{appendix:ipad_complex_missing}

\section{Performance under Combined Feature and Edge Missing Conditions}
To further assess the robustness of the proposed \textbf{\framework} framework, an additional experiment was conducted under complex extreme missing data scenarios, where both node features and graph structure were heavily corrupted. Specifically, the evaluation considered a uniform feature missing rate of 99.5\% in conjunction with an edge missing rate of 60\%, simulating an exceptionally challenging setting for graph learning. Results are summarized in Table~\ref{tab:mix_missing_u}.

Across four benchmark datasets (Computers, Photo, WikiCS, Physics), \textbf{\framework}-enhanced models were compared against their corresponding SOTA counterparts augmented with standard feature propagation (FP) preprocessing. While the accuracy gap between iPaD and FP-based baselines was reduced under such severe dual corruption, \textit{iPaD} still maintained competitive performance levels, particularly for the GCN backbone, which achieved an average accuracy of 78.43\% compared to 79.41\% for its FP-based variant.

It is worth noting that, unlike the single \textbf{edge missing} scenario reported in Section~\ref{sec:exp_edge_missing} where the \textbf{\framework} architecture consistently outperformed TunedGNN, such a performance advantage was not observed under the combined corruption setting. The primary reason lies in the fact that \textbf{edge missing} also degrades the performance of the imputer module, which in turn limits the student model’s ability to identify and exploit reliable feature signals to replace suboptimal teacher model information. As a result, the overall performance becomes more dependent on the teacher model’s reliability, leading to a reduced advantage for the distilled student.

These findings suggest that, although extreme simultaneous feature and edge removal significantly challenges all evaluated methods, \textbf{\framework} retains a comparable level of predictive capability while preserving its inherent efficiency advantages. This reinforces the framework’s applicability in scenarios where graph data is subject to multi-faceted degradation.

% 定義底色
\definecolor{lightgray}{gray}{0.95}
\definecolor{blue1}{RGB}{0,128,255}
\definecolor{orange1}{RGB}{235,120,23}
\definecolor{green1}{RGB}{20,170,70}
\definecolor{mygray}{gray}{0.825}


\begingroup
\rowcolors{2}{mygray}{white}


\renewcommand{\arraystretch}{1.1} % 增加行高,表格更寬鬆



\begin{table}[H]
     \centering
    \begin{tabular}{L{0.265\textwidth}|llll|C{0.11\textwidth}}
\textbf{Model} & \textbf{\small Computer} & \textbf{\small Photo} & \textbf{\small WikiCS} & \textbf{\small Physics} & \textbf{\small Average} \textbf{\footnotesize Accuracy}\\

         \midrule
        {\small SGFormer w FP} & 81.82 {\footnotesize ± 0.81} & 86.81 {\footnotesize ± 0.62} & 63.61 {\footnotesize ± 0.33} & 83.01 {\footnotesize ± 0.67} & 78.13\\
        {\small Polynormer w FP} & 80.67 {\footnotesize ± 4.66} & 85.01 {\footnotesize ± 2.21} & 61.38 {\footnotesize ± 0.58} & 77.82 {\footnotesize ± 4.49} & 76.22\\

        \midrule
        {\small TunedGCN w FP} & 83.54 {\footnotesize ± 0.49} & 88.44 {\footnotesize ± 0.49} & 62.06 {\footnotesize ± 0.27} & 83.59 {\footnotesize ± 0.13} & 79.41\\
        \textbf{\small \framework-GCN(Our)} & 82.88 {\footnotesize ± 0.76} & 83.69 {\footnotesize ± 0.41} & 63.43 {\footnotesize ± 0.38} & 83.61 {\footnotesize ± 0.10} & 78.43\\

        \midrule
        {\small TunedSAGE w FP} & 85.74 {\footnotesize ± 0.58} & 88.46 {\footnotesize ± 0.37} & 63.00 {\footnotesize ± 0.34} & 82.75 {\footnotesize ± 0.08} & 80.00\\
        \textbf{\small \framework-SAGE(Our)} & 80.47 {\footnotesize ± 0.12} & 83.69 {\footnotesize ± 0.41} & 63.43 {\footnotesize ± 0.38} & 82.81 {\footnotesize ± 0.72} & 77.61\\
        \bottomrule
    \end{tabular}
    \caption{\textbf{\small Performance under complex extreme missing data conditions.}
Node classification accuracy (\%) on four datasets (Computer, Photo, WikiCS, Physics) with uniform feature missing rate of 99.5\% and edge missing rate of 60\%.
Shaded rows indicate our iPaD variants, compared against SOTA models with feature propagation (FP) preprocessing.}
    \label{tab:mix_missing_u}
\end{table}
\endgroup







