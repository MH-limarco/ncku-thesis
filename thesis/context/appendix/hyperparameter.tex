\StartChapter{Hyperparameters}{appendix:hyperparameters}

This appendix provides a detailed analysis of the key hyperparameters for our proposed modules, justifying the settings used in our main experiments. To ensure a systematic and unbiased selection, we utilized the \textbf{Optuna}~\cite{akiba2019optuna} to perform automated hyperparameter optimization (HPO). For each major experiment, we ran over 100 trials, optimizing for the validation set accuracy. The following sections present the insights gained from this process.

\section{Analysis of APCFI Hyperparameters}
\label{sec:appendix_apcfi}

For our \textbf{APCFI} module, we analyzed the two most sensitive hyperparameters: the confidence decay factor \texbf{alpha} and the inter-channel propagation strength \texbf{beta}.
The contour plot in Figure~\ref{fig:contour_apcfi} visualizes their complex interaction.

\InsertFigure
    [caption={Contour plot for the key hyperparameters (\texbf{alpha} and \texbf{beta}) of the APCFI module on the Cora dataset.},
    label={fig:contour_apcfi},
    scale=0.5]
    {./context/fig/cora_apcfi_contour_plot.png}
    %\centering
    %\includegraphics[width=0.8\linewidth]{./context/fig/cora_kd_contour_plot.png}
    %\caption{Contour plot for the key hyperparameters (`alpha` and `beta`) of the APCFI module on the Cora dataset.}
    %\label{fig:contour_apcfi}
%\end{figure}

The key insight from this plot is the \textbf{strong interaction between alpha and beta}.
The optimal value for one parameter is highly dependent on the value of the other, leading to several \textbf{islands of high performance}.
For instance, when \textbf{beta} is very low (around $10^{-5}$), a lower \textbf{alpha} (around 0.2-0.4) can achieve good results.
However, the largest and most stable high-performance region occurs when \textbf{alpha is in a mid-to-high range (0.5 to 0.8)} and \textbf{beta is in a low-to-mid range (around $10^{-3}$)}.
This analysis demonstrates that while the module is sensitive to this parameter combination, its behavior is predictable and optimizable, and it provides a clear guideline for selecting a robust set of default parameters.


\section{Analysis of MP-KRD Hyperparameters}
\label{sec:appendix_mpkrd}

For our \textbf{MP-KRD} module, we identified \textbf{alpha}, \textbf{lamb}, and \textbf{momentum} as three critical hyperparameters influencing the distillation curriculum and training dynamics. Figure~\ref{fig:contour_mpkrd} shows the contour plot illustrating the pairwise interactions of these parameters on the final model accuracy.

\begin{figure}[htbp]
    \centering
    \includegraphics[width=0.8\linewidth]{./context/fig/cora_kd_contour_plot.png}
    \caption{Contour plot for the key hyperparameters of the MP-KRD module. Darker blue regions indicate higher accuracy. The plot reveals strong dependencies between parameters.}
    \label{fig:contour_mpkrd}
\end{figure}

The analysis reveals several clear trends.
The most decisive parameter is \textbf{momentum}, where higher values (approximately 0.85 to 1.0) consistently correlate with superior performance.
Conversely, the \textbf{lamb} parameter performs best in a lower range (approximately 0 to 2.5).
The impact of \textbf{alpha} is more subtle and interacts with the other two.
Overall, the plot indicates that the optimal performance \textbf{sweet spot} is achieved with a combination of \textbf{high momentum}, \textbf{low lamb}, and \textbf{mid-to-low alpha}.
This data-driven insight guided our final parameter selection for the MP-KRD experiments.

