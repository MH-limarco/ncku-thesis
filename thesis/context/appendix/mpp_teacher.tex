\StartChapter{Effect of MPP Pruning on Teacher Models}{appendix:mpp_teacher}

\section{Rationale for Applying MPP Pruning to Teacher Models}
To improve the efficiency of the distillation process, the teacher models are first pruned using the proposed Multi-Path Proxy (MPP) strategy prior to knowledge transfer.
While parameter pruning inherently reduces the model size, an appropriately configured MPP preserves the most informative channels and removes redundant computations, thereby maintaining the predictive capability of the teacher.
As shown in Table~\ref{tab:pruning_comparison_medium}, the pruned teacher models exhibit negligible or no accuracy degradation compared to their unpruned counterparts, while delivering substantial improvements in training speed (e.g., from 19.35 to 54.49 epoch/s for TunedGCN).
Furthermore, the MPP strategy is not limited to a specific backbone and can be effectively extended to various classic or state-of-the-art (SOTA) architectures, such as SGFormer, Polynormer, GCN, and GraphSAGE, consistently providing both training efficiency gains and competiti

% 定義底色
\definecolor{lightgray}{gray}{0.95}
\definecolor{blue1}{RGB}{0,128,255}
\definecolor{orange1}{RGB}{235,120,23}
\definecolor{green1}{RGB}{20,170,70}
\definecolor{mygray}{gray}{0.825}


\begingroup
\rowcolors{3}{white}{mygray}


\renewcommand{\arraystretch}{1.1} % 增加行高,表格更寬鬆



\begin{table}[H]
     \centering
    \begin{tabular}{L{0.265\textwidth}|L{0.15\textwidth}L{0.15\textwidth}L{0.15\textwidth}L{0.15\textwidth}|C{0.11\textwidth}}
\textbf{} & \textbf{\small Computers} & \textbf{\small Photo} & \textbf{\small WikiCS} & \textbf{\small CS}\\
  \textbf{\small Model} & \textbf{\footnotesize Accuracy} \textbf{\scriptsize (Train(epcoh/s))} &  \textbf{\footnotesize Accuracy} \textbf{\scriptsize (Train(epcoh/s))}& \textbf{\footnotesize Accuracy} \textbf{\scriptsize (Train(epcoh/s))} &   \textbf{\footnotesize Accuracy} \textbf{\scriptsize (Train(epcoh/s))} & \textbf{\small Average} \textbf{\footnotesize Accuracy}\\

         \midrule
         {\small SGFormer} & 92.00 \textbf{\scriptsize (58.05)} & 96.34 \textbf{\scriptsize (75.48)} & 80.05 \textbf{\scriptsize (27.22)} & 95.31 \textbf{\scriptsize (25.57)} & 90.93\\
        \textbf{\small MPP-SGFormer(Our)} & \textbf{91.87} \textbf{\scriptsize (73.00)} & \textbf{95.46}  \textbf{\scriptsize (81.12)}& \textbf{78.81}  \textbf{\scriptsize (45.99)}& \textbf{95.15}  \textbf{\scriptsize (39.81)} &  \textbf{90.32}\\

         \midrule
        {\small Polynormer} & 93.00 \textbf{\scriptsize (6.90)} & 96.31 \textbf{\scriptsize (9.54)} &  79.66 \textbf{\scriptsize (6.19)} & 94.81 \textbf{\scriptsize (8.72)} & 90.95\\
        \textbf{\small MPP-Polynormer(Our)} & \textbf{93.22} \textbf{\scriptsize (15.94)} & \textbf{96.33} \textbf{\scriptsize (19.10)} & \textbf{79.79}  \textbf{\scriptsize (14.58)} & \textbf{93.08}  \textbf{\scriptsize (19.34)} & \textbf{90.61}\\

        \midrule
        {\small TunedGCN} & 93.64 \textbf{\scriptsize (19.35)} & 95.97 \textbf{\scriptsize (28.84)} & 80.43 \textbf{\scriptsize (41.64)} & 95.89 \textbf{\scriptsize (23.32)} & 91.48\\
        \textbf{\small MPP-TunedGCN(Our)} & \textbf{93.62} \textbf{\scriptsize (54.49)} & \textbf{95.92} \textbf{\scriptsize (47.20)} & \textbf{80.68}  \textbf{\scriptsize (69.58)} & \textbf{96.01}  \textbf{\scriptsize (47.73)} & \textbf{91.56}\\
        \midrule
        {\small TunedSAGE} & 93.14 \textbf{\scriptsize (55.97)} & 96.04 \textbf{\scriptsize (73.37)} & 80.01  \textbf{\scriptsize (89.41)} & 95.94 \textbf{\scriptsize (15.71)} & 91.28\\
        \textbf{\small MPP-TunedSAGE(Our)} & \textbf{92.87} \textbf{\scriptsize (90.49)} & \textbf{95.83} \textbf{\scriptsize (90.49)} & \textbf{80.27}  \textbf{\scriptsize (113.37)} & \textbf{95.87}  \textbf{\scriptsize (21.29)} & \textbf{91.21}\\
        \bottomrule
    \end{tabular}
    \caption{\textbf{\small Training efficiency improvement of MPP pruning over various SOTA models.}
For each dataset (Computers, Photo, WikiCS, CS), accuracy (\%) is reported alongside training speed (epoch/s) in parentheses. }
    \label{tab:pruning_comparison_medium}
\end{table}
\endgroup




